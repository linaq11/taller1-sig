% Options for packages loaded elsewhere
% Options for packages loaded elsewhere
\PassOptionsToPackage{unicode}{hyperref}
\PassOptionsToPackage{hyphens}{url}
\PassOptionsToPackage{dvipsnames,svgnames,x11names}{xcolor}
%
\documentclass[
  letterpaper,
  DIV=11,
  numbers=noendperiod]{scrartcl}
\usepackage{xcolor}
\usepackage{amsmath,amssymb}
\setcounter{secnumdepth}{-\maxdimen} % remove section numbering
\usepackage{iftex}
\ifPDFTeX
  \usepackage[T1]{fontenc}
  \usepackage[utf8]{inputenc}
  \usepackage{textcomp} % provide euro and other symbols
\else % if luatex or xetex
  \usepackage{unicode-math} % this also loads fontspec
  \defaultfontfeatures{Scale=MatchLowercase}
  \defaultfontfeatures[\rmfamily]{Ligatures=TeX,Scale=1}
\fi
\usepackage{lmodern}
\ifPDFTeX\else
  % xetex/luatex font selection
\fi
% Use upquote if available, for straight quotes in verbatim environments
\IfFileExists{upquote.sty}{\usepackage{upquote}}{}
\IfFileExists{microtype.sty}{% use microtype if available
  \usepackage[]{microtype}
  \UseMicrotypeSet[protrusion]{basicmath} % disable protrusion for tt fonts
}{}
\makeatletter
\@ifundefined{KOMAClassName}{% if non-KOMA class
  \IfFileExists{parskip.sty}{%
    \usepackage{parskip}
  }{% else
    \setlength{\parindent}{0pt}
    \setlength{\parskip}{6pt plus 2pt minus 1pt}}
}{% if KOMA class
  \KOMAoptions{parskip=half}}
\makeatother
% Make \paragraph and \subparagraph free-standing
\makeatletter
\ifx\paragraph\undefined\else
  \let\oldparagraph\paragraph
  \renewcommand{\paragraph}{
    \@ifstar
      \xxxParagraphStar
      \xxxParagraphNoStar
  }
  \newcommand{\xxxParagraphStar}[1]{\oldparagraph*{#1}\mbox{}}
  \newcommand{\xxxParagraphNoStar}[1]{\oldparagraph{#1}\mbox{}}
\fi
\ifx\subparagraph\undefined\else
  \let\oldsubparagraph\subparagraph
  \renewcommand{\subparagraph}{
    \@ifstar
      \xxxSubParagraphStar
      \xxxSubParagraphNoStar
  }
  \newcommand{\xxxSubParagraphStar}[1]{\oldsubparagraph*{#1}\mbox{}}
  \newcommand{\xxxSubParagraphNoStar}[1]{\oldsubparagraph{#1}\mbox{}}
\fi
\makeatother


\usepackage{longtable,booktabs,array}
\usepackage{calc} % for calculating minipage widths
% Correct order of tables after \paragraph or \subparagraph
\usepackage{etoolbox}
\makeatletter
\patchcmd\longtable{\par}{\if@noskipsec\mbox{}\fi\par}{}{}
\makeatother
% Allow footnotes in longtable head/foot
\IfFileExists{footnotehyper.sty}{\usepackage{footnotehyper}}{\usepackage{footnote}}
\makesavenoteenv{longtable}
\usepackage{graphicx}
\makeatletter
\newsavebox\pandoc@box
\newcommand*\pandocbounded[1]{% scales image to fit in text height/width
  \sbox\pandoc@box{#1}%
  \Gscale@div\@tempa{\textheight}{\dimexpr\ht\pandoc@box+\dp\pandoc@box\relax}%
  \Gscale@div\@tempb{\linewidth}{\wd\pandoc@box}%
  \ifdim\@tempb\p@<\@tempa\p@\let\@tempa\@tempb\fi% select the smaller of both
  \ifdim\@tempa\p@<\p@\scalebox{\@tempa}{\usebox\pandoc@box}%
  \else\usebox{\pandoc@box}%
  \fi%
}
% Set default figure placement to htbp
\def\fps@figure{htbp}
\makeatother





\setlength{\emergencystretch}{3em} % prevent overfull lines

\providecommand{\tightlist}{%
  \setlength{\itemsep}{0pt}\setlength{\parskip}{0pt}}



 


\KOMAoption{captions}{tableheading}
\makeatletter
\@ifpackageloaded{caption}{}{\usepackage{caption}}
\AtBeginDocument{%
\ifdefined\contentsname
  \renewcommand*\contentsname{Table of contents}
\else
  \newcommand\contentsname{Table of contents}
\fi
\ifdefined\listfigurename
  \renewcommand*\listfigurename{List of Figures}
\else
  \newcommand\listfigurename{List of Figures}
\fi
\ifdefined\listtablename
  \renewcommand*\listtablename{List of Tables}
\else
  \newcommand\listtablename{List of Tables}
\fi
\ifdefined\figurename
  \renewcommand*\figurename{Figure}
\else
  \newcommand\figurename{Figure}
\fi
\ifdefined\tablename
  \renewcommand*\tablename{Table}
\else
  \newcommand\tablename{Table}
\fi
}
\@ifpackageloaded{float}{}{\usepackage{float}}
\floatstyle{ruled}
\@ifundefined{c@chapter}{\newfloat{codelisting}{h}{lop}}{\newfloat{codelisting}{h}{lop}[chapter]}
\floatname{codelisting}{Listing}
\newcommand*\listoflistings{\listof{codelisting}{List of Listings}}
\makeatother
\makeatletter
\makeatother
\makeatletter
\@ifpackageloaded{caption}{}{\usepackage{caption}}
\@ifpackageloaded{subcaption}{}{\usepackage{subcaption}}
\makeatother
\usepackage{bookmark}
\IfFileExists{xurl.sty}{\usepackage{xurl}}{} % add URL line breaks if available
\urlstyle{same}
\hypersetup{
  pdftitle={Taller 1: Evaluación Comparativa de Procesamiento Geoespacial},
  pdfauthor={Lina Quintero},
  colorlinks=true,
  linkcolor={blue},
  filecolor={Maroon},
  citecolor={Blue},
  urlcolor={Blue},
  pdfcreator={LaTeX via pandoc}}


\title{Taller 1: Evaluación Comparativa de Procesamiento Geoespacial}
\author{Lina Quintero}
\date{2026-02-15}
\begin{document}
\maketitle


\subsection{Metodología}\label{metodologuxeda}

En este ejercicio se procesó una banda de Sentinel-2A
(\textasciitilde120 millones de píxeles) para comparar el rendimiento de
diferentes motores geoespaciales bajo una operación aritmética y una
reducción global (media).

\subsection{Parte A: Resultados en
JupyterLab}\label{parte-a-resultados-en-jupyterlab}

\begin{longtable}[]{@{}llll@{}}
\toprule\noalign{}
Motor Geoespacial & Lenguaje & Tiempo (s) & Media Global \\
\midrule\noalign{}
\endhead
\bottomrule\noalign{}
\endlastfoot
\textbf{R: terra} & R (C++) & 10.672 & 3766.625 \\
\textbf{R: stars} & R & 14.675 & 3766.625 \\
\textbf{Python: rasterio} & Python & 1.587 & 766.624630 \\
\textbf{Julia: Rasters.jl} & Julia & 7.10847472 & 3766.62463 \\
\end{longtable}

\subsubsection{Observaciones por
Notebook}\label{observaciones-por-notebook}

\begin{itemize}
\tightlist
\item
  \textbf{01\_benchmark\_terra.ipynb}: El motor \texttt{terra} es
  altamente eficiente ya que gestiona los archivos por referencia y
  realiza los cálculos en C++.
\item
  \textbf{02\_benchmark\_stars.ipynb}: Este motor suele ser más lento en
  reducciones globales debido a la materialización de datos en la
  memoria de R.
\item
  \textbf{03\_benchmark\_rasterio.ipynb}: Destaca por su velocidad al
  utilizar arreglos de NumPy optimizados con instrucciones de hardware.
\item
  \textbf{04\_benchmark\_rasters\_julia.ipynb}: Julia utiliza un modelo
  de pipelines composables que, tras la compilación inicial, permite un
  procesamiento nativo muy fluido.
\end{itemize}

\subsection{parte B --- Ejecución desde VSCode (terminal
integrada)}\label{parte-b-ejecuciuxf3n-desde-vscode-terminal-integrada}

\subsection{Comparación de Rendimiento: Notebooks
vs.~Terminal}\label{comparaciuxf3n-de-rendimiento-notebooks-vs.-terminal}

\begin{longtable}[]{@{}
  >{\raggedright\arraybackslash}p{(\linewidth - 4\tabcolsep) * \real{0.3077}}
  >{\centering\arraybackslash}p{(\linewidth - 4\tabcolsep) * \real{0.3462}}
  >{\centering\arraybackslash}p{(\linewidth - 4\tabcolsep) * \real{0.3462}}@{}}
\toprule\noalign{}
\begin{minipage}[b]{\linewidth}\raggedright
Motor de Procesamiento
\end{minipage} & \begin{minipage}[b]{\linewidth}\centering
Tiempo Notebook (Parte A)
\end{minipage} & \begin{minipage}[b]{\linewidth}\centering
Tiempo Terminal (Parte B)
\end{minipage} \\
\midrule\noalign{}
\endhead
\bottomrule\noalign{}
\endlastfoot
\textbf{R: terra} & 10.672 s & 12.736 s \\
\textbf{R: stars} & 14.675 s & 15.227 s \\
\textbf{Python: rasterio} & 1.587 s & 2.621 s \\
\textbf{Julia: Rasters.jl} & 7.108 s & 13.457 s \\
\end{longtable}

\subsection{Parte C --- Ejecución desde el Termina de Windows
(PowerShell)}\label{parte-c-ejecuciuxf3n-desde-el-termina-de-windows-powershell}

\subsubsection{Resumen Comparativo de Tiempos de
Procesamiento}\label{resumen-comparativo-de-tiempos-de-procesamiento}

\begin{longtable}[]{@{}
  >{\raggedright\arraybackslash}p{(\linewidth - 6\tabcolsep) * \real{0.2162}}
  >{\centering\arraybackslash}p{(\linewidth - 6\tabcolsep) * \real{0.2252}}
  >{\centering\arraybackslash}p{(\linewidth - 6\tabcolsep) * \real{0.2703}}
  >{\centering\arraybackslash}p{(\linewidth - 6\tabcolsep) * \real{0.2883}}@{}}
\toprule\noalign{}
\begin{minipage}[b]{\linewidth}\raggedright
Motor de Procesamiento
\end{minipage} & \begin{minipage}[b]{\linewidth}\centering
Parte A: JupyterLab (s)
\end{minipage} & \begin{minipage}[b]{\linewidth}\centering
Parte B: Terminal VS Code (s)
\end{minipage} & \begin{minipage}[b]{\linewidth}\centering
Parte C: PowerShell Directo (s)
\end{minipage} \\
\midrule\noalign{}
\endhead
\bottomrule\noalign{}
\endlastfoot
\textbf{R: terra} & 10.672 & 12.736 & 15.368 \\
\textbf{Python: rasterio} & 1.587 & 2.621 & 11.06 \\
\textbf{R: stars} & 14.675 & 2.621 s & 2.477 \\
\textbf{Julia: Rasters.jl} & 7.108 & 13.457 s & 18.36 \\
\end{longtable}

\textbf{Nota sobre la Parte C:} Para la ejecución desde PowerShell se
utilizó la \textbf{Opción 2 (Ejecución directa)} mediante el comando
\texttt{docker\ exec}. Se observa que este método es el más eficiente
para automatizar procesos de GeoAI, ya que consume menos recursos de
interfaz gráfica de los 12GB de RAM asignados al contenedor.

\subsubsection{Diferencias Notadas}\label{diferencias-notadas}

\begin{itemize}
\item
  \textbf{Optimización de RAM:} Al prescindir de la interfaz de VS Code
  o Jupyter, se liberan recursos del sistema, permitiendo que los
  \textbf{12GB de RAM} se enfoquen exclusivamente en el procesamiento de
  datos.
\item
  \textbf{Latencia de Inicialización:} La ejecución en PowerShell
  presenta una ligera demora inicial comparada con Jupyter, ya que debe
  cargar el intérprete y las librerías en cada llamada, mientras que en
  el Notebook el kernel ya está activo.
\item
  \textbf{Estabilidad:} La terminal de Windows ofrece una ejecución más
  robusta ante bloqueos de la interfaz web, ideal para procesos que
  requieren alta intensidad de cómputo.
\item
  \textbf{Compilación JIT:} En Julia, la terminal permite monitorear de
  forma más limpia el proceso de compilación inicial antes de la
  ejecución del benchmark.
\end{itemize}

\subsection{Análisis y Conclusiones
Finales}\label{anuxe1lisis-y-conclusiones-finales}

\subsection{1. 📍 Entorno de ejecución}\label{entorno-de-ejecuciuxf3n}

¿Notaron diferencias de tiempo entre JupyterLab, VSCode y PowerShell?
Sí, se observaron variaciones. Los Notebooks (JupyterLab) suelen ser más
rápidos en ejecuciones repetitivas porque el kernel ya está activo y las
librerías cargadas en los 12GB de RAM.

Razón técnica: La ejecución en PowerShell o Terminal tiene un mayor
overhead inicial, ya que cada vez que se corre un script, el sistema
debe inicializar el intérprete (R, Python o Julia) y cargar todos los
paquetes desde el disco al contenedor. En cambio, el kernel de Jupyter
mantiene el entorno ``caliente''.

\subsection{2. 🧱 Abstracción en la
práctica}\label{abstracciuxf3n-en-la-pruxe1ctica}

¿En qué motor creen que el costo de las abstracciones es más visible? En
R (específicamente con stars).

\subsection{Relación con los
tiempos:}\label{relaciuxf3n-con-los-tiempos}

Al observar los tiempos de 14.675 s en Jupyter, se nota que, aunque la
sintaxis es muy amigable (alta abstracción), el motor realiza múltiples
conversiones internas y manejo de metadatos que lo hacen más lento que
terra o rasterio. La abstracción facilita el código pero penaliza la
velocidad de procesamiento puro.

\subsection{3. 🔥 Julia y el costo de compilación
(Warm-up)}\label{julia-y-el-costo-de-compilaciuxf3n-warm-up}

¿El efecto del warm-up de Julia se notó más en algún entorno específico?
Se notó mucho más en la Terminal (VS Code y PowerShell).

En la terminal, Julia debe realizar la compilación JIT (Just-In-Time) de
las librerías Rasters y ArchGDAL en cada ejecución individual del
script. En JupyterLab, una vez que la primera celda compila los
paquetes, las ejecuciones siguientes son casi instantáneas porque el
código ya está traducido a lenguaje de máquina en la memoria activa.

\subsection{4. 🧠 Elección informada}\label{elecciuxf3n-informada}

¿Cambiarían su elección del ``Titán'' para una emergencia ambiental
real? No, mantendría a Python (rasterio) como el titán por su velocidad
consistente (cerca de 1.5s - 2.6s) en todos los entornos. En una
emergencia, la capacidad de ejecutar scripts rápidos vía terminal
(PowerShell) sin depender de una interfaz pesada es vital para procesar
grandes volúmenes de datos Sentinel-2 de manera automática y confiable
bajo la configuración de 12GB de RAM optimizada.




\end{document}
